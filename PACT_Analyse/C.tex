\section{Context}
Intro shit
\subsection{Physical}
In regards to physical constraints that the application might have, we have to consider a quite important aspect: Whether the application is run on a non-static portable device like a touch tablet or smartphone, or if it's run on a personal computer in the users own home office. If the application is portable there is a lot of physical issues that derives from the medium itself, such as:
\begin{itemize}[noitemsep]
    \item Rain, sun glare heavy wind or other external weather hazards that might damage or destroy the device on which the program is being executed.
    \item Distracting the user while driving or doing other important tasks that will require attention.
    \item Power issues and Internet access out on the field.
    \item If the application is being used on a home PC, there is much less issues with Weather and distractions, but the user might want to use the application while his/her memory is fresh.
\end{itemize}

\subsection{Social}
The farmer needs to learn to use the program, before he can draw advantage from it. The user might not have great IT-skills, so the learning curve should be gentle.

\subsection{Organisational}
\begin{itemize}[noitemsep]
    \item The program will enable the user to maximize profits, while minimizing expenses to hiring help, because an estimate for the amount of work required will be available way before it is needed, and therefore give the opportunity to optimize the hiring procedure.
    \item The program will also enable the user to get a better overview of the economy of his field(s), because he will be able to estimate expenses on each field separately and decide if the current approach is sufficiently profitable.
    \item Savings will be possible when buying crops and equipment, because the user will be able to find offers in advance, instead of having to find something in the last minute, during the harvest/sowing/plowing period.
\end{itemize}
