\textbf{Fuel efficiency} \newline
In this section we will elaborate on the fuel efficiency of tractors and combine harvesters.
[METATEKST]\todo{write metatext for this}

Tractors
In our studies of the subject we've found that there are many different sizes of tractors, physically as well as in terms of power, and have therefore decided to split them up into three subclasses:
\begin{itemize}[noitemsep]
    \item small tractors (less than 150 bhp)
    \item medium tractors (150-250 bhp)
    \item large tractors (more than 250 bhp)
\end{itemize}
Small tractors have a fuel usage with an average of roughly 20 litres/hour (calculated average of 3 tractors in multiple use scenarios \cite{dlg:case_maxxum_130}\cite{dlg:john_deere_6125R}\cite{dlg:fendt-313-scr}), while medium-sized tractors use roughly 29 litres/hour\cite{dlg:fendt-724-scr}\cite{dlg:claas_arion_650}\cite{dlg:claas_axion_850} and large tractors use 48.5 litres/hour on average\cite{dlg:fendt_939}\cite{dlg:john_deere_8335}\cite{dlg:claas_axion_950}.
Combine harvesters has been more difficult to find exact measures on, we haven't been able to find any specific measures from an independent source, nor was the information available on the 3 manufacturers websites (Claas, Case IH and Massey Ferguson), that we checked, so we will not be able to give an expected average fuel consumption for these.
Another source \cite{erfarland:dieselforbrug} claims that the average field requires 90-100 litres of diesel per hectar in average, and can range from 65-120 litres in the extremes, for an entire season.
Based on the large difference in fuel consumption, found in our research on tractors affected by, amongst others, load, the quality of the surface being driven on, the engine size/power, and the type of work being done, we assume that the fuel economy of combine harvesters will vary in a similar degree. This means that it will be impossible to calculate a single average to be used in lieu of input of historical data from previous years. We have therefore concluded that it will be more useful to allow the user to input their own expected average fuel consumption, if they do not have sufficient data available for the specific field. 
