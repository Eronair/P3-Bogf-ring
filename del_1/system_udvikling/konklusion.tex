Through our research in the above sections, we have established a baseline on what a theoretical solution, based on user requirements, might be. To further work towards an end solution, a summation and a look into technological advances in the field is needed, as to better understand the current advances within our problem domain.

Our system definitions act as the establishment of a problem domain, and can be regarded as a rough estimate on what a possible design choice could be. 
The interview has given us an initial list of requirements, that through the use of PACT-analysis can be shortened to the following requirements in the upcoming chapters:

\begin{itemize}
    \item 
\end{itemize}

The context section of the PACT-analysis allowed us to identify external factors that could influence the use of the application, such as weather conditions and internet connectivity.



We have defined our problem domain thanks to our initial contact with the user. The system definitions allowed us to observe the problem domain from different perspectives. After the second interview we found that the first system definition, which focused primarily on being a monitoring and recording tool, was the one that our primary source found most appealing and useful.
Our knowledge from the final system definition was the foundation for our PACT-analysis. This PACT-analysis allowed us to identify the users and their their capabilities and skills as well as several design considerations. One of the considerations is that the users will use the application regularly most of the year, which means that the user will develop a degree of familiarity with the application, thus lowering the demands for immediate usability. The context section of the PACT-analysis allowed us to identify external factors that could influence the use of the application, such as weather conditions and internet connectivity. Lastly the technology section has made it possible to determine design limitations and requirements, such as input- and output-devices.

what we are going to take from this part are the design considerations as described above. As well as the requriemnts given to us by the interview and the technological limitations that a tablet forces us to consider.
The second interview revealed system requirements for the solution, along with identifying the problems our primary source hoped to resolve.



%We can conclude that after the second interview [??] we found the system definition that our primary source found most appealing and useful.

% Defining User Requirements
%  - initial contact and wishes from user
%  - FACTOR-model
% Interview <- analysis of
%  - program requirements
%  - problems
% PACT-Analysis
%  - People
%  - Activities
%  - Context
%  - Technology

%----------------------------------------------------------------------
% Pact analyse 
%   - afsløret ting vi skal lave/undersøge videre
% System definitions
%   - Har enabled os til at se vores problem domain
% Samlet set har alt dette ladet os konstruere en problemstilling bla bla bla.
% Ud fra vores system definition, interview og analyse kan vi konkludere at $manden har et problem, vi har en løsning, der skal samarbejdes og så er alle glade.