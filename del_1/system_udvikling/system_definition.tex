Throughout this section of the report, we will be focusing on defining a system, within the requirements stated by our user. This is done to better understand what the contact is looking for in a functional application, and to allow ourselves better chances of adhering to those requirements, through the establishment of a problem domain.


As stated in our interview in section \ref{interview_analysis}, is that at the current time, our user feels that he has trouble keeping track of the information about the field that he is tending with his company.
The information that the user wishes to have at his disposal is:

\begin{itemize}
    \item Resource usage
    \item Labor usage
    \item Machine usage
    \item Combined expenses and revenue
\end{itemize}

Our user wishes to have all this combined on a tablet, as to have easy access to all the information that he would need, in the field. As such, we have modelled three different system definitions, that our user has to choose from, to better understand what solution the user wants, and needs, in a functional program.

The first system definition consists of a computerized system that monitors, collects and records information from different sources such as GPS, and other farming applications. The system should primarily be a monitoring and recording tool, and secondarily be a tool and medium to figure out the cost of labor and resources used in a farming job. The system should be based on a tablet with current tools. The system should be able to function outside in a rough environment, and be used by farming professionals with varying degree of technical experience, but who are expected to be trained in the use of the system. 

Taking this into a FACTOR-model, we can see that our solution would fulfill the requirements stated by our user:

\begin{itemize}[noitemsep]
    \item \textbf{F}unctionality: Monitor, collect and record information from different farming applications.
    \item \textbf{A}pplication domain: Workers of the farm, billing authority.
    \item \textbf{C}onditions: Outside in rough environment used by trained professionals of varying technical skill.
    \item \textbf{T}echnology: Handheld device such as a tablet computer or smartphone.
    \item \textbf{O}bjects: Information from farming applications.
    \item \textbf{R}esponsibility: Monitoring and recording tool, cost calculator and cost justification.
\end{itemize}

Our second system definition bases it self around a computerized system used as a planning and tracking tool for farmers that can register and track field dimensions, GPS, tractor fuel consumption and other related information. And applies that information as a resource to plot out the most effective harvesting route as well as providing real-time information to the user during use. Executed on a handheld device, the system should primarily serve as a planning tool, and secondarily double as a real-time guidance tool, that assists the farmer to follow his previously set plan. The system should be able to function in rough weather and be usable for users of variable technical experience.

As seen by the FACTOR-model below, the second system also meets the users requirements, but with a different approach:

\begin{itemize}[noitemsep]
    \item \textbf{F}unctionality: Maximize harvesting output.
    \item \textbf{A}pplication domain: Workers in the agricultural industry.
    \item \textbf{C}onditions: Outside in rough weather and be usable by users of variable technical skill.
    \item \textbf{T}echnology: Handheld device such as a tablet computer or smartphone.
    \item \textbf{O}bjects: Field dimensions and tractor fuel consumption.
    \item \textbf{R}esponsibility: Planning and guidance tool for farm tracking.
\end{itemize}

Our third and final system definition, has a fundamentally different approach to supervising fields. Instead of monitoring the machines used on the field, it will monitor the field itself. Through sensors dug into the soil, that periodically transmit data about the conditions in the ground, such as the amount of nutrients and water. This data will then be collected whenever the farmer is nearby with his tablet, and then allow him to use this knowledge to optimize sowing, fertilizing and harvesting. If combined with weather forecasts, it will furthermore be able to give a prediction of the future situation, in a limited time span. The system will be used by farmers, and the sensors need to be very sturdy, because they will have to survive outside in all kinds of weather. The tablet with the application will still need to be somewhat sturdy, but does not have as rigorous and extreme demands, since the farmer can choose to delay his trip to the field if the weather is very bad. 

\begin{itemize}[noitemsep]
    \item \textbf{F}unctionality: Aid in determining the correct time of various work-related actions on a field.
    \item \textbf{A}pplication domain: Workers in the agricultural industry. Can also be used in greenhouses, professional or not.
    \item \textbf{C}onditions: Usable outside, used by workers with training in the program.
    \item \textbf{T}echnology: Tablet with wireless connectivity
    \item \textbf{O}bjects: Information on soil conditions.
    \item \textbf{R}esponsibility: Presenting data about soil conditions in a comprehensible way.
\end{itemize}

\todo{skriv Jens' valg, og en begrænsning til afsnittet.}
Hence we can define a problem domain as follows:

The problem-domain is the field, the resources, labor and machine requirements.

The application-domain is the agricultural contractor and his machines.

We will use these system definitions when interviewing our primary source in the section below. \todo{sammenhængende tekst ?} 
